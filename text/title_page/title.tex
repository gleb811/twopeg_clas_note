\noindent\begin{minipage}{\textwidth}
\begin{center}
\thispagestyle{empty}
\vspace{0.5cm}
{ \Large{TWOPEG: An Event Generator for Charged Double Pion Electroproduction off Proton}}\\
\vspace{1cm}

{\large Iu. Skorodumina$^{1a, 3a}$, G.V. Fedotov$^{4b, 1b}$, V.D. Burkert$^{2}$, E. Golovach$^{4}$,\\ R.W. Gothe$^{1}$, V. Mokeev$^{2}$}\\[6pt]

\parbox{.86\textwidth}{\centering\small\it
$^1$ Department of Physics and Astronomy, University of South Carolina, 712 Main Street, Columbia, SC 29208, USA.\\
$^2$ Thomas Jefferson National Accelerator Facility, 12000 Jefferson Avenue, Newport News, VA 23606.\\
$^3$ Faculty of Physics, Moscow State University, Moscow, 119991, Russia\\
$^4$ Skobel'tsyn Institute of Nuclear Physics, Moscow State University, Moscow, 119991, Russia\\
E-mail: $^a$ skorodum@jlab.org, $^b$ gleb@jlab.org}\\


\vspace{2cm}
{\bf Abstract}\\[9pt]

\end{center}
{The new event generator TWOPEG for the channel $e p \rightarrow e' p' \pi^{+} \pi^{-}$ has been developed. It uses an advanced method of event generation with weights and employs the five-fold differential structure functions from the recent versions of the JM model fit to all results on charged double pion photo- and electroproduction cross sections from CLAS (both published and preliminary). In the areas covered by measured CLAS data, TWOPEG successfully reproduces the available integrated and single-differential double pion cross sections. To estimate the cross sections in the regions not covered by data, a specialized extrapolation procedure is applied. The EG currently covers a kinematical area in $Q^2$ starting from 0.0005 GeV$^2$ and in $W$ from the reaction threshold up to 4.5 GeV. TWOPEG allows to obtain the cross section values from the generated distributions and simulates radiative effects. The link to the code is provided. TWOPEG has already been used in CLAS data analyses and in PAC proposal preparations and is designed to be used during the CLAS12 era.}

%\vspace{1pt}\par


\end{minipage}
